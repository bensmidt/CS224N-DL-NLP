\documentclass[12pt]{article}
\usepackage{lingmacros}
\usepackage{tree-dvips}
% hyper links
\usepackage[utf8]{inputenc}
\usepackage{amsmath}
\usepackage{amsfonts}
\usepackage{bbm}
\usepackage{bm}
% Formatting quotes properly
\usepackage[english]{babel}
\usepackage[autostyle, english = american]{csquotes}
\MakeOuterQuote{"}
\usepackage{hyperref}
\usepackage{enumitem}
\usepackage{array}
    %\setlength{\extrarowheight}{1pt}
\usepackage{lipsum}



% \newif\ifanswers
% \answerstrue % comment out to hide answers

% \usepackage[compact]{titlesec}
% \usepackage{fancyhdr} % Required for custom headers
% \usepackage{lastpage} % Required to determine the last page for the footer
% \usepackage{extramarks} % Required for headers and footers
% \usepackage[usenames,dvipsnames]{color} % Required for custom colors
% \usepackage{graphicx} % Required to insert images
% \usepackage{listings} % Required for insertion of code
% \usepackage{courier} % Required for the courier font
% \usepackage{lipsum} % Used for inserting dummy 'Lorem ipsum' text into the template
% \usepackage{enumerate}
% \usepackage{enumitem}
% \usepackage{subfigure}
% \usepackage{booktabs}
% \usepackage{amsmath, amsthm, amssymb}
% \usepackage{caption}
% \usepackage{hyperref}
% \captionsetup[table]{skip=4pt}
% \usepackage{framed}
% \usepackage{bm}
% \usepackage{minted}
% \usepackage{tikz}
% \usetikzlibrary{positioning,patterns,fit}

\begin{document}

\noindent Author: Benjamin Smidt

\noindent Created: October 17th, 2022

\noindent Last Updated: October 28th, 2022
\begin{center}
\section*{CS 224N A3: Dependency Parsing}
\end{center}

\paragraph{} Note to the reader. This is my work for assignment two of Stanford's course
\href{https://web.stanford.edu/class/cs224n/}{CS 224N: Natural Language Processing with Deep Learning}. 
You can find the lecture Winter 2021 lectures series on YouTube 
\href{https://www.youtube.com/playlist?list=PLoROMvodv4rOSH4v6133s9LFPRHjEmbmJ}{here}.
This document is meant to be used as a reference, explanation, and resource for the assignment, 
not necessarily a comprehensive overview of Word Vectors. If there's a typo or a correction 
needs to be made, feel free to email me at benjamin.smidt@utexas.edu so I can fix it. 
Thank you! I hope you find this document helpful :). 

\tableofcontents{}

\newpage

\section{Machine Learning and Neural Networks}

\subsection{Adam Optimizer}
In our traditional Stochastic Gradient Descent, the update rule is 
\begin{equation*}
    \theta \leftarrow \theta - \alpha \nabla_{\theta} J_{\text{minibatch}}(\theta)
\end{equation*}
The Adam optimizer modifies SGD such in an effort to improve convergence. The first 
addition is the use of \emph{momentum}. Adam keeps a rolling average of the gradients 
instead of using only the current gradient. 
\begin{equation}
    m \leftarrow \beta_1 m + (1 - \beta_1) \nabla_\theta J_{\text{minibatch}} (\theta)
\end{equation}
\begin{equation*}
    \theta \leftarrow \theta - \alpha m
\end{equation*}
(i) \emph{Briefly explain in 2-4 sentences how using $m$ stops the updates from 
varying as much and why this low variance may be helpful to learning, overall.}
~\\
~\\
$m$ is a weighted average between all the previous updates, embedded in $m$, and the current 
update $\nabla_\theta J_{\text{minibatch}}(\theta)$ (our $\beta_1$ parameter specifies the 
weight to give each term, $\beta_1 = 0.9$ is common). By keeping this weighted average, 
the update naturally gives higher weight to updating in directions that have been 
consistent while updates along dimensions that keep switching between positive and negative 
are given close to no weight. This improves optimization since our updates will minimize steps 
in dimensions that aren't getting us anywhere meaningful (flipping between positive and negative, 
can't decide which direction to go in) and maximize steps in dimensions that are getting us 
somewhere meaningful (nearly all updates have had this direction). 
~\\
~\\
A second addition to Adam is \emph{adaptive learning rates}, which keeps track of $v$, a rolling average
of the magnitude of the gradients. 
\begin{equation*}
    m \leftarrow \beta_1 m + (1 - \beta_1) \nabla_\theta J_{\text{minibatch}}(\theta)
\end{equation*}
\begin{equation*}
    v \leftarrow \beta_2 v + (1 - \beta_2) (\nabla_\theta J_{\text{minibatch}}(\theta) 
    \; \odot \; \nabla_\theta J_{\text{minibatch}}(\theta) )
\end{equation*}
\begin{equation*}
    \theta \leftarrow \theta - \alpha m / \sqrt{v}
\end{equation*}
$odot$ is elementwise multiplication and $/$ is elementwise division. $\beta_2$ is our second 
hyperparameter (often set to 0.99). 
~\\
~\\
(ii) \emph{Since Adam divides the update by $\sqrt{v}$, which of the model parameter will get larger updates? 
Why might this help with learning?}
~\\
~\\
If $v$ is quite small, then dividing by $\sqrt{v}$ will make the term $\alpha m / \sqrt{v}$ large.
This improve learning because often times we need a large learning rate if our gradient update is naturally
small. 
~\\
~\\
The vice versa is also true. When the gradient is very large ($v$ is very large), then we don't
need a very large learning rate and often a small learning rate will be better. In this 
case, by dividing by $\sqrt{v}$, we actually reduce the magnitude of the update to counterbalance
the already large gradient. 

\subsection{Dropout}

Dropout is a form of regularization wherein we "drop" random connections within the 
hidden layers of our network during each update (different connections are 
dropped for each update). We do this mathematically with the following
\begin{equation*}
    h_{\text{drop}} = \gamma d \odot h 
\end{equation*}
where $h$ is a hidden layer, $d \in \{0, 1\}^{D_h}$ ($D_h$ is the size of $h$) is a mask vector
with each entry being 0 (with probability $p_{\text{drop}}$) or 1 (with probability 
$1 - p_{\text{drop}}$), and $\gamma$ is a constant chosen such that the expected 
value of $h_{\text{drop}}$ is $h$
\begin{equation*}
    \mathbb{E}_{p_{\text{drop}}} [h_{\text{drop}}]_i = h_i \; \; \; \forall \; i \in \{1, \dotso, D_h\}
\end{equation*}
(i) \emph{What must $\gamma$ equal in terms of $p_{\text{drop}}$? Briefly justify your answer or 
show your math derivation using the equations given above}

\begin{equation*}
    \mathbb{E}_{p_{\text{drop}}} [h_{\text{drop}}] = h
\end{equation*}
\begin{equation*}
    \mathbb{E}_{p_{\text{drop}}} [\gamma d \odot h ] = h
\end{equation*}
\begin{equation*}
    \gamma \; \mathbb{E}_{p_{\text{drop}}} [d \odot h] = h
\end{equation*}
\begin{equation*}
    \gamma [h p_{\text{drop}} + (1 - p_{\text{drop}})0] = h
\end{equation*}
\begin{equation*}
    \gamma \; h p_{\text{drop}} = h
\end{equation*}
\begin{equation*}
    \gamma = \frac{1}{p_{\text{drop}}}
\end{equation*}

(ii) \emph{Why should dropout be applied during training? Why should dropout 
NOT be applied during evaluation?}
~\\
~\\
Dropout should be applied during training so the network learns different
pathways that lead to the same prediction. By closing different connections 
randomly, the network is forced to produce multiple paths in which data can flow 
to achieve the correct prediction, theoretically making it more robust. 
~\\
~\\
We wouldn't want to apply dropout during evaluation however because our 
results would be non-deterministic. Due to the randomness of the 
dopout connections, it's possible (and may even be likely depending on the 
network) that evaluating the same input twice yields different predictions. 
Obviously this is an undesirable trait to have in a machine learning model 
so we don't apply dropout during evaluation. 


\section{Neural Transition-Based Dependency Parsing}

\subsection{Problem A}

\bigskip
\scriptsize
\addtolength{\tabcolsep}{-1pt}
\begin{tabular}{l|l|c|c}
    Stack & Buffer & New Dependency & Transition \\
    \hline
    ROOT & I, parsed, this sentence, correctly && Initial Configuration \\
    ROOT, I & parsed, this sentence, correctly && SHIFT \\
    ROOT, I, parsed & this sentence, correctly && SHIFT \\
    ROOT, parsed & this sentence, correctly & parsed $\rightarrow$ I & LEFT-ARC \\
    ROOT, parsed, this & sentence, correctly  && SHIFT \\
    ROOT, parsed, this, sentence & correctly && SHIFT \\
    ROOT, parsed, sentence & correctly & sentence $\rightarrow$ this & LEFT-ARC \\
    ROOT, parsed & correctly & parsed $\rightarrow$ sentence & RIGHT-ARC \\
    ROOT, parsed, correctly &&& SHIFT \\
    ROOT, parsed && parsed $\rightarrow$ correctly & RIGHT-ARC \\
    ROOT && ROOT $\rightarrow$ parsed & RIGHT-ARC \\
\end{tabular}

\normalsize
\subsection{Problem B}
There are only two options at a given step: push a word onto the stack or pop 
a word (and create a dependency) off of the stack. Since every word must be 
pushed onto the stack a single time and popped off the stack a single time, 
this leaves us with $2n$ steps. This is concurrent with our table which has 
11 rows: $2(5) + 1$ where $n = 5$ and we have an additional row for our 
initial state of the stack with only ROOT. 

\subsection{Problem C: Init and Parse Step}
I'm not going to go over the code in any details because using implementing 
what we just talked about is pretty trivial (you can see the code yourself 
if you want to see how its done). The only important thing to remember is 
that we do NOT want to modify the sentence we're parsing so we must make 
a COPY of the sentence when initializing the buffer. Otherwise, the buffer 
will point to the same memory location as the sentence and we'll be modifying 
our original sentence, which we don't want. 

\subsection{Problem D: Minibatch}
Kind of the same deal as part C, the pseudo code is given so we just have to work out 
the actual implementation which isn't too difficult, particularly since 
we're using Python. Just see the code for how this is done, the instructions 
explain high level what's happening already. 

\subsection{Problem E: Training and Test}
I know this seems a bit lazy maybe, but I'm not going to explain this seciton either. 
I spent a tremendous amount of time learning and explaining all of this 
in the EECS 498: Deep Learning for Computer Vision course. The details aren't the 
same (obviously), but building neural networks, using PyTorch, optimizing, etc. 
are all covered in DEPTH in the assignments (for which I have written very detailed 
explanations to accompany them). 

sThus, I will suffice to say that you need only look 
at the code for this assignment to understand what I'm doing (provided you understand 
DL). If you don't, I'd recommend the EECS 498: Deep Learning for Computer Vision as it 
is the best introduction to Deep Learning I've found so far (even if it is a little 
"older" (2019)). 

\subsection{}



\end{document}